%!TEX root = Humanoids2013.tex
\section{Conclusions and future work}

In this paper, we have studied the role of soft actuation in the reduction of the energy cost for mechanical systems that perform cyclic tasks. We have considered both SEAs and PEAs and determined the optimal stiffness value and spring pre-load such that a given cost functional is minimized. We have shown that the energy consumption also depends on the shape of the joint trajectories used to perform a given cyclic task. In this paper the desired joint trajectories are sinusoids which may not guarantee the best behavior in terms of energy saving. Future works will be dedicated to explore the role of the shape of the joint trajectories in the reduction of the energy cost as well as how to design them.

Moreover, results obtained in this paper can be applied directly to more complex systems such as hopping or humanoid robots for which soft actuation can be exploited to achieve a reduction of the the so called ``Cost of Transport'' which is an important aspect of these robots. Future works will be dedicated to show by experimentation the reduction of the energy consumption.

\section{Acknowledgment}
Authors gratefully acknowledge Felipe Belo, Fabio Bonomo, Manuel Giuseppe Catalano, Manuel Bonilla, Kamilo Melo, Alessandro Serio and Andrea Di Basco for their assistance in the prototype design.

